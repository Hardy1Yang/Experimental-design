% Options for packages loaded elsewhere
\PassOptionsToPackage{unicode}{hyperref}
\PassOptionsToPackage{hyphens}{url}
%
\documentclass[
]{article}
\usepackage{lmodern}
\usepackage{amssymb,amsmath}
\usepackage{ifxetex,ifluatex}
\ifnum 0\ifxetex 1\fi\ifluatex 1\fi=0 % if pdftex
  \usepackage[T1]{fontenc}
  \usepackage[utf8]{inputenc}
  \usepackage{textcomp} % provide euro and other symbols
\else % if luatex or xetex
  \usepackage{unicode-math}
  \defaultfontfeatures{Scale=MatchLowercase}
  \defaultfontfeatures[\rmfamily]{Ligatures=TeX,Scale=1}
\fi
% Use upquote if available, for straight quotes in verbatim environments
\IfFileExists{upquote.sty}{\usepackage{upquote}}{}
\IfFileExists{microtype.sty}{% use microtype if available
  \usepackage[]{microtype}
  \UseMicrotypeSet[protrusion]{basicmath} % disable protrusion for tt fonts
}{}
\makeatletter
\@ifundefined{KOMAClassName}{% if non-KOMA class
  \IfFileExists{parskip.sty}{%
    \usepackage{parskip}
  }{% else
    \setlength{\parindent}{0pt}
    \setlength{\parskip}{6pt plus 2pt minus 1pt}}
}{% if KOMA class
  \KOMAoptions{parskip=half}}
\makeatother
\usepackage{xcolor}
\IfFileExists{xurl.sty}{\usepackage{xurl}}{} % add URL line breaks if available
\IfFileExists{bookmark.sty}{\usepackage{bookmark}}{\usepackage{hyperref}}
\hypersetup{
  pdftitle={實驗設計作業一},
  pdfauthor={楊昊紘},
  hidelinks,
  pdfcreator={LaTeX via pandoc}}
\urlstyle{same} % disable monospaced font for URLs
\usepackage[margin=1in]{geometry}
\usepackage{color}
\usepackage{fancyvrb}
\newcommand{\VerbBar}{|}
\newcommand{\VERB}{\Verb[commandchars=\\\{\}]}
\DefineVerbatimEnvironment{Highlighting}{Verbatim}{commandchars=\\\{\}}
% Add ',fontsize=\small' for more characters per line
\usepackage{framed}
\definecolor{shadecolor}{RGB}{248,248,248}
\newenvironment{Shaded}{\begin{snugshade}}{\end{snugshade}}
\newcommand{\AlertTok}[1]{\textcolor[rgb]{0.94,0.16,0.16}{#1}}
\newcommand{\AnnotationTok}[1]{\textcolor[rgb]{0.56,0.35,0.01}{\textbf{\textit{#1}}}}
\newcommand{\AttributeTok}[1]{\textcolor[rgb]{0.77,0.63,0.00}{#1}}
\newcommand{\BaseNTok}[1]{\textcolor[rgb]{0.00,0.00,0.81}{#1}}
\newcommand{\BuiltInTok}[1]{#1}
\newcommand{\CharTok}[1]{\textcolor[rgb]{0.31,0.60,0.02}{#1}}
\newcommand{\CommentTok}[1]{\textcolor[rgb]{0.56,0.35,0.01}{\textit{#1}}}
\newcommand{\CommentVarTok}[1]{\textcolor[rgb]{0.56,0.35,0.01}{\textbf{\textit{#1}}}}
\newcommand{\ConstantTok}[1]{\textcolor[rgb]{0.00,0.00,0.00}{#1}}
\newcommand{\ControlFlowTok}[1]{\textcolor[rgb]{0.13,0.29,0.53}{\textbf{#1}}}
\newcommand{\DataTypeTok}[1]{\textcolor[rgb]{0.13,0.29,0.53}{#1}}
\newcommand{\DecValTok}[1]{\textcolor[rgb]{0.00,0.00,0.81}{#1}}
\newcommand{\DocumentationTok}[1]{\textcolor[rgb]{0.56,0.35,0.01}{\textbf{\textit{#1}}}}
\newcommand{\ErrorTok}[1]{\textcolor[rgb]{0.64,0.00,0.00}{\textbf{#1}}}
\newcommand{\ExtensionTok}[1]{#1}
\newcommand{\FloatTok}[1]{\textcolor[rgb]{0.00,0.00,0.81}{#1}}
\newcommand{\FunctionTok}[1]{\textcolor[rgb]{0.00,0.00,0.00}{#1}}
\newcommand{\ImportTok}[1]{#1}
\newcommand{\InformationTok}[1]{\textcolor[rgb]{0.56,0.35,0.01}{\textbf{\textit{#1}}}}
\newcommand{\KeywordTok}[1]{\textcolor[rgb]{0.13,0.29,0.53}{\textbf{#1}}}
\newcommand{\NormalTok}[1]{#1}
\newcommand{\OperatorTok}[1]{\textcolor[rgb]{0.81,0.36,0.00}{\textbf{#1}}}
\newcommand{\OtherTok}[1]{\textcolor[rgb]{0.56,0.35,0.01}{#1}}
\newcommand{\PreprocessorTok}[1]{\textcolor[rgb]{0.56,0.35,0.01}{\textit{#1}}}
\newcommand{\RegionMarkerTok}[1]{#1}
\newcommand{\SpecialCharTok}[1]{\textcolor[rgb]{0.00,0.00,0.00}{#1}}
\newcommand{\SpecialStringTok}[1]{\textcolor[rgb]{0.31,0.60,0.02}{#1}}
\newcommand{\StringTok}[1]{\textcolor[rgb]{0.31,0.60,0.02}{#1}}
\newcommand{\VariableTok}[1]{\textcolor[rgb]{0.00,0.00,0.00}{#1}}
\newcommand{\VerbatimStringTok}[1]{\textcolor[rgb]{0.31,0.60,0.02}{#1}}
\newcommand{\WarningTok}[1]{\textcolor[rgb]{0.56,0.35,0.01}{\textbf{\textit{#1}}}}
\usepackage{graphicx,grffile}
\makeatletter
\def\maxwidth{\ifdim\Gin@nat@width>\linewidth\linewidth\else\Gin@nat@width\fi}
\def\maxheight{\ifdim\Gin@nat@height>\textheight\textheight\else\Gin@nat@height\fi}
\makeatother
% Scale images if necessary, so that they will not overflow the page
% margins by default, and it is still possible to overwrite the defaults
% using explicit options in \includegraphics[width, height, ...]{}
\setkeys{Gin}{width=\maxwidth,height=\maxheight,keepaspectratio}
% Set default figure placement to htbp
\makeatletter
\def\fps@figure{htbp}
\makeatother
\setlength{\emergencystretch}{3em} % prevent overfull lines
\providecommand{\tightlist}{%
  \setlength{\itemsep}{0pt}\setlength{\parskip}{0pt}}
\setcounter{secnumdepth}{-\maxdimen} % remove section numbering

\title{實驗設計作業一}
\author{楊昊紘}
\date{2020/10/22}

\begin{document}
\maketitle

2.3 In your own words, describe what is meant by the terms (a) grand
mean, (b) treatment effect,(c) error effect

2.9 The following data on running time (in seconds) in a straight-alley
maze were obtained in a CRF-33 design.

\begin{Shaded}
\begin{Highlighting}[]
\NormalTok{A=}\KeywordTok{c}\NormalTok{(}\StringTok{"a1"}\NormalTok{,}\StringTok{"a1"}\NormalTok{,}\StringTok{"a1"}\NormalTok{,}\StringTok{"a2"}\NormalTok{,}\StringTok{"a2"}\NormalTok{,}\StringTok{"a2"}\NormalTok{,}\StringTok{"a3"}\NormalTok{,}\StringTok{"a3"}\NormalTok{,}\StringTok{"a3"}\NormalTok{)}
\NormalTok{B=}\KeywordTok{c}\NormalTok{(}\StringTok{"b1"}\NormalTok{,}\StringTok{"b2"}\NormalTok{,}\StringTok{"b3"}\NormalTok{,}\StringTok{"b1"}\NormalTok{,}\StringTok{"b2"}\NormalTok{,}\StringTok{"b3"}\NormalTok{,}\StringTok{"b1"}\NormalTok{,}\StringTok{"b2"}\NormalTok{,}\StringTok{"b3"}\NormalTok{)}
\NormalTok{Y=}\KeywordTok{c}\NormalTok{(}\DecValTok{9}\NormalTok{,}\DecValTok{8}\NormalTok{,}\DecValTok{5}\NormalTok{,}\DecValTok{7}\NormalTok{,}\DecValTok{7}\NormalTok{,}\DecValTok{5}\NormalTok{,}\DecValTok{6}\NormalTok{,}\DecValTok{5}\NormalTok{,}\DecValTok{5}\NormalTok{)}
\NormalTok{Data=}\KeywordTok{data.frame}\NormalTok{(Y,A,B)}

\KeywordTok{ggplot}\NormalTok{(Data,}\KeywordTok{aes}\NormalTok{(}\DataTypeTok{x=}\NormalTok{A,}\DataTypeTok{y=}\NormalTok{Y,}\DataTypeTok{color=}\NormalTok{B))}\OperatorTok{+}\KeywordTok{geom_line}\NormalTok{(}\KeywordTok{aes}\NormalTok{(}\DataTypeTok{group=}\NormalTok{B))}\OperatorTok{+}\KeywordTok{geom_point}\NormalTok{(}\DataTypeTok{size=}\DecValTok{5}\NormalTok{)}
\end{Highlighting}
\end{Shaded}

\includegraphics{HW1_files/figure-latex/unnamed-chunk-1-1.pdf}

\begin{Shaded}
\begin{Highlighting}[]
\KeywordTok{ggplot}\NormalTok{(Data,}\KeywordTok{aes}\NormalTok{(}\DataTypeTok{x=}\NormalTok{B,}\DataTypeTok{y=}\NormalTok{Y,}\DataTypeTok{color=}\NormalTok{A))}\OperatorTok{+}\KeywordTok{geom_line}\NormalTok{(}\KeywordTok{aes}\NormalTok{(}\DataTypeTok{group=}\NormalTok{A))}\OperatorTok{+}\KeywordTok{geom_point}\NormalTok{(}\DataTypeTok{size=}\DecValTok{5}\NormalTok{)}
\end{Highlighting}
\end{Shaded}

\includegraphics{HW1_files/figure-latex/unnamed-chunk-1-2.pdf}

Graph the interaction. Give a verbal description of the interaction.

2.13 Distinguish among the following concepts. (a) Sample distribution,
population distribution, and sampling distribution (b)Sample statistic
and test statistic

2.16 Indicate the type of error or correct decision for each of the
following. a.A true null hypothesis was rejected. b.The researcher
failed to reject a false null hypothesis. c.The null hypothesis is false
and the researcher rejected it. d.The researcher did not reject a true
null hypothesis. e.A false null hypothesis was rejected. f.The
researcher rejected the null hypothesis when he or she should have
failed to reject it.

2.21 What advantages do confidence interval procedures have over null
hypothesis-testing procedures?

3.3 For each of the following chi-square variables, determine the mean
and variance.

\begin{enumerate}
\def\labelenumi{\alph{enumi}.}
\item
\end{enumerate}

3.15 Determine the degrees of freedom for SSTO, SSBG, and SSWG for the
following CRp designs. CR-3 with n = 10 CR-4 with n1 = 3, n2 = 4, n3 =
4, n4 = 5

3.18 Explain why a test of is a test of the hypothesis that α1 = α2 =
\ldots{} = αP = 0.

3.21 Discuss the statement, ``The F test is not appropriate for
dichotomous data because such data depart radically from the normal
distribution.

4.5 An experiment was designed to evaluate the effects of different
levels of training on children's ability to acquire the concept of an
equilateral triangle. Fifty 3-year-old children were recruited from
daycare facilities and randomly assigned to one of five groups, with 10
children in each group. Each group contained an equal number of boys and
girls. Children in treatment level a1 (visual condition) were shown 36
blocks, one at a time, and instructed to look at them but not to touch
them. Children in treatment level a2 (visual plus motor condition)
looked at the blocks and were permitted to play with them. They also
were asked to perform specific tactile-kinesthetic exercises, such as
tracing the perimeter of the blocks with their index finger. Children in
treatment level a3 (visual plus verbal condition) looked at the blocks
and were told to notice differences in their shape, color, size, and
thickness. Children in treatment level a4 (visual plus motor plus verbal
condition) used a combination of visual, motor, and verbal means of
stimulus predifferentiation. Children in treatment level a5 (control
condition) engaged in unrelated play activity. All training was done
individually. The day after training, the children were shown a
``target'' block for 5 seconds and then asked to identify the block in a
group of seven blocks. This task was repeated six times using different
target blocks. The dependent variable was the number of target blocks
correctly identified. The following data were obtained. (Experiment
suggested by Nelson, G. K. Concomitant effects of visual, motor, and
verbal experiences in young children's concept development. Journal of
Educational Psychology.)

a.Perform an exploratory data analysis on these data (see Table 4.2-1
and Figure 4.2-1). Assume that the observations within each treatment
level are listed in the order in which the observations were obtained.
Interpret the analysis.

b.Test the null hypothesis μ1 = μ2 = \ldots{} = μ5; let α = .05.
Construct an ANOVA table and make a decision about the null hypothesis.

c.Compute and interpret and for these data. e.Use the results of part
(b) as a pilot study and determine the number of subjects required to
achieve a power of approximately .80. f.Use Appendix Table E.12 to
determine the number of subjects required to detect a medium
association; let 1 -- β = .80. g.Determine the number of subjects
required to achieve a power of .80, where the largest difference among
means is 0.95σε.

\begin{Shaded}
\begin{Highlighting}[]
\NormalTok{a1=}\KeywordTok{c}\NormalTok{(}\DecValTok{0}\NormalTok{,}\DecValTok{1}\NormalTok{,}\DecValTok{3}\NormalTok{,}\DecValTok{1}\NormalTok{,}\DecValTok{1}\NormalTok{,}\DecValTok{2}\NormalTok{,}\DecValTok{2}\NormalTok{,}\DecValTok{1}\NormalTok{,}\DecValTok{1}\NormalTok{,}\DecValTok{2}\NormalTok{)}
\NormalTok{a2=}\KeywordTok{c}\NormalTok{(}\DecValTok{2}\NormalTok{,}\DecValTok{3}\NormalTok{,}\DecValTok{4}\NormalTok{,}\DecValTok{2}\NormalTok{,}\DecValTok{1}\NormalTok{,}\DecValTok{1}\NormalTok{,}\DecValTok{2}\NormalTok{,}\DecValTok{2}\NormalTok{,}\DecValTok{3}\NormalTok{,}\DecValTok{4}\NormalTok{)}
\NormalTok{a3=}\KeywordTok{c}\NormalTok{(}\DecValTok{2}\NormalTok{,}\DecValTok{3}\NormalTok{,}\DecValTok{4}\NormalTok{,}\DecValTok{4}\NormalTok{,}\DecValTok{2}\NormalTok{,}\DecValTok{1}\NormalTok{,}\DecValTok{2}\NormalTok{,}\DecValTok{3}\NormalTok{,}\DecValTok{2}\NormalTok{,}\DecValTok{2}\NormalTok{)}
\NormalTok{a4=}\KeywordTok{c}\NormalTok{(}\DecValTok{2}\NormalTok{,}\DecValTok{4}\NormalTok{,}\DecValTok{5}\NormalTok{,}\DecValTok{3}\NormalTok{,}\DecValTok{2}\NormalTok{,}\DecValTok{1}\NormalTok{,}\DecValTok{3}\NormalTok{,}\DecValTok{3}\NormalTok{,}\DecValTok{2}\NormalTok{,}\DecValTok{4}\NormalTok{)}
\NormalTok{a5=}\KeywordTok{c}\NormalTok{(}\DecValTok{1}\NormalTok{,}\DecValTok{0}\NormalTok{,}\DecValTok{2}\NormalTok{,}\DecValTok{1}\NormalTok{,}\DecValTok{1}\NormalTok{,}\DecValTok{2}\NormalTok{,}\DecValTok{1}\NormalTok{,}\DecValTok{0}\NormalTok{,}\DecValTok{1}\NormalTok{,}\DecValTok{3}\NormalTok{)}
\NormalTok{order=}\KeywordTok{rep}\NormalTok{(}\KeywordTok{c}\NormalTok{(}\DecValTok{1}\OperatorTok{:}\DecValTok{10}\NormalTok{),}\KeywordTok{rep}\NormalTok{(}\DecValTok{5}\NormalTok{,}\DecValTok{10}\NormalTok{))}
\NormalTok{Dta=}\KeywordTok{data.frame}\NormalTok{(a1,a2,a3,a4,a5)}
\NormalTok{Dta.long=}\KeywordTok{pivot_longer}\NormalTok{(Dta,}\DataTypeTok{cols=}\KeywordTok{c}\NormalTok{(a1,a2,a3,a4,a5),}\DataTypeTok{names_to =} \StringTok{"learning"}\NormalTok{)}
\NormalTok{Dta.long=}\KeywordTok{mutate}\NormalTok{(Dta.long,order)}
\NormalTok{model1=}\KeywordTok{lm}\NormalTok{(value}\OperatorTok{~}\NormalTok{learning,Dta.long)}
\NormalTok{Aov=}\KeywordTok{anova}\NormalTok{(model1)}
\end{Highlighting}
\end{Shaded}

\hypertarget{including-plots}{%
\subsection{Including Plots}\label{including-plots}}

\begin{Shaded}
\begin{Highlighting}[]
\NormalTok{omega_sq=(Aov}\OperatorTok{$}\StringTok{`}\DataTypeTok{Sum Sq}\StringTok{`}\NormalTok{[}\DecValTok{1}\NormalTok{]}\OperatorTok{-}\NormalTok{(}\DecValTok{5-1}\NormalTok{)}\OperatorTok{*}\NormalTok{Aov}\OperatorTok{$}\StringTok{`}\DataTypeTok{Mean Sq}\StringTok{`}\NormalTok{[}\DecValTok{2}\NormalTok{])}\OperatorTok{/}\NormalTok{(}\KeywordTok{sum}\NormalTok{(Aov}\OperatorTok{$}\StringTok{`}\DataTypeTok{Sum Sq}\StringTok{`}\NormalTok{)}\OperatorTok{+}\NormalTok{Aov}\OperatorTok{$}\StringTok{`}\DataTypeTok{Mean Sq}\StringTok{`}\NormalTok{[}\DecValTok{2}\NormalTok{])}
\NormalTok{f=}\KeywordTok{sqrt}\NormalTok{(omega_sq}\OperatorTok{/}\NormalTok{(}\DecValTok{1}\OperatorTok{-}\NormalTok{omega_sq))}
\end{Highlighting}
\end{Shaded}

\begin{Shaded}
\begin{Highlighting}[]
\NormalTok{model.st=}\KeywordTok{rstandard}\NormalTok{(model1)}
\NormalTok{Dta.long=}\KeywordTok{mutate}\NormalTok{(Dta.long,model.st)}
\NormalTok{Dta.wide=}\KeywordTok{pivot_wider}\NormalTok{(Dta.long,}\DataTypeTok{names_from=}\NormalTok{learning,}\DataTypeTok{values_from =}\NormalTok{ value)}
\NormalTok{  p1=}\KeywordTok{ggplot}\NormalTok{(}\KeywordTok{filter}\NormalTok{(Dta.long,learning}\OperatorTok{==}\StringTok{"a1"}\NormalTok{), }\KeywordTok{aes}\NormalTok{(}\DataTypeTok{x=}\NormalTok{order, }\DataTypeTok{y=}\NormalTok{model.st, }\DataTypeTok{color=}\NormalTok{learning)) }\OperatorTok{+}
\StringTok{  }\KeywordTok{geom_line}\NormalTok{()}\OperatorTok{+}
\StringTok{  }\KeywordTok{geom_point}\NormalTok{() }

\NormalTok{  p2=}\KeywordTok{ggplot}\NormalTok{(}\KeywordTok{filter}\NormalTok{(Dta.long,learning}\OperatorTok{==}\StringTok{"a2"}\NormalTok{), }\KeywordTok{aes}\NormalTok{(}\DataTypeTok{x=}\NormalTok{order, }\DataTypeTok{y=}\NormalTok{model.st, }\DataTypeTok{color=}\NormalTok{learning)) }\OperatorTok{+}
\StringTok{  }\KeywordTok{geom_line}\NormalTok{()}\OperatorTok{+}
\StringTok{  }\KeywordTok{geom_point}\NormalTok{() }
\NormalTok{  p3=}\KeywordTok{ggplot}\NormalTok{(}\KeywordTok{filter}\NormalTok{(Dta.long,learning}\OperatorTok{==}\StringTok{"a3"}\NormalTok{), }\KeywordTok{aes}\NormalTok{(}\DataTypeTok{x=}\NormalTok{order, }\DataTypeTok{y=}\NormalTok{model.st, }\DataTypeTok{color=}\NormalTok{learning)) }\OperatorTok{+}
\StringTok{  }\KeywordTok{geom_line}\NormalTok{()}\OperatorTok{+}
\StringTok{  }\KeywordTok{geom_point}\NormalTok{() }
\NormalTok{  p4=}\KeywordTok{ggplot}\NormalTok{(}\KeywordTok{filter}\NormalTok{(Dta.long,learning}\OperatorTok{==}\StringTok{"a4"}\NormalTok{), }\KeywordTok{aes}\NormalTok{(}\DataTypeTok{x=}\NormalTok{order, }\DataTypeTok{y=}\NormalTok{model.st, }\DataTypeTok{color=}\NormalTok{learning)) }\OperatorTok{+}
\StringTok{  }\KeywordTok{geom_line}\NormalTok{()}\OperatorTok{+}
\StringTok{  }\KeywordTok{geom_point}\NormalTok{() }
\NormalTok{  p5=}\KeywordTok{ggplot}\NormalTok{(}\KeywordTok{filter}\NormalTok{(Dta.long,learning}\OperatorTok{==}\StringTok{"a5"}\NormalTok{), }\KeywordTok{aes}\NormalTok{(}\DataTypeTok{x=}\NormalTok{order, }\DataTypeTok{y=}\NormalTok{model.st, }\DataTypeTok{color=}\NormalTok{learning)) }\OperatorTok{+}
\StringTok{  }\KeywordTok{geom_line}\NormalTok{()}\OperatorTok{+}
\StringTok{  }\KeywordTok{geom_point}\NormalTok{() }

\KeywordTok{grid.arrange}\NormalTok{(p1,p2,p3,p4,p5,}\DataTypeTok{nrow=}\DecValTok{2}\NormalTok{)}
\end{Highlighting}
\end{Shaded}

\includegraphics{HW1_files/figure-latex/unnamed-chunk-4-1.pdf}

\begin{Shaded}
\begin{Highlighting}[]
\KeywordTok{print}\NormalTok{(Aov)}
\end{Highlighting}
\end{Shaded}

\begin{verbatim}
## Analysis of Variance Table
## 
## Response: value
##           Df Sum Sq Mean Sq F value   Pr(>F)   
## learning   4  21.88  5.4700  5.3745 0.001264 **
## Residuals 45  45.80  1.0178                    
## ---
## Signif. codes:  0 '***' 0.001 '**' 0.01 '*' 0.05 '.' 0.1 ' ' 1
\end{verbatim}

\end{document}
